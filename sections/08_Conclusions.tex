%%%%%%%%%%%%%%%%%%%%%%%%%%%%%%%%%%%%%%%%%%%%%%%%%%
%% Authors :    Giovanni Arcari                 %%
%%              Maria Bronzova                  %%
%%              Luca Sardi                      %%
%%              Spencer Sharp                   %%
%%                                              %%
%% Supervisor : Andreas Apostolatos             %%
%%											  	%%
%% e-mail : andreas.apostolatos@tum.de		   	%%
%%											  	%%
%% 08_Conclusions.tex					  	   	%%
%%											  	%%
%%%%%%%%%%%%%%%%%%%%%%%%%%%%%%%%%%%%%%%%%%%%%%%%%%
\section{CONCLUSIONS}
Sensitivity analysis is a powerful tool in the field of Structural Design. In general, it quantifies the effect of input variables on output variables of an objective function. There are three basic categories of sensitivity analyses: global, semi-analytical, and analytical. All three were implemented in Matlab, utilizing the provided finite element code as a basis. Three different objective functions were implemented: strain energy, displacement, and Von Mises stress. In addition, the GiD user interface was enhanced to allow the user to input point loads, customize his/her desired sensitivity analysis, and input parameters needed for optimization. This information is passed fluidly to Matlab, where the implemented code will generate an output file that can be visualized in Paraview.
