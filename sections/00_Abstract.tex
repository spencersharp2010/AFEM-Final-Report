%%%%%%%%%%%%%%%%%%%%%%%%%%%%%%%%%%%%%%%%%%%%%%%%%%
%% Authors :    Giovanni Arcari                 %%
%%              Mariia Bronzova                  %%
%%              Luca Sardi                      %%
%%              Spencer Sharp                   %%
%%                                              %%
%% Supervisor : Andreas Apostolatos             %%
%%											  	%%
%% e-mail : andreas.apostolatos@tum.de		   	%%
%%											  	%%
%% 00_Abstract.tex					  	   	    %%
%%											  	%%
%%%%%%%%%%%%%%%%%%%%%%%%%%%%%%%%%%%%%%%%%%%%%%%%%%
\section*{\normalsize{ABSTRACT}}

%\pagenumbering{arabic}

\textit{The goal of the assignment is to develop and run sensitivity analyses for different objective functions for a 2-dimensional linear elasticity problem with standard finite elements. Finite element code was provided as a basis on which to develop the sensitivity analysis. Three different types of sensitivity analyses were implemented - global, semi-analytical, and analytical - for three different objective functions - strain energy, displacement, and Von Mises stress.  As an input for the calculations, a set of nodal displacement is taken; the results illustrate how they affect the analyzed objective functions. To perform each calculation, finite differencing methods are used. For all of the analyses, a single reference case, in terms of geometry, loading, boundary conditions, and material properties, is considered. A convergence study was performed in order to ensure a proper perturbation value was chosen for the nodal displacements. The code was validated by a comparison of results from the various types of sensitivity analyses that were implemented. Finally, the sensitivity analysis was utilized to perform a relatively simple optimization of the given geometry.}\\

\paragraph*{Keywords : sensitivity analysis, optimization, finite differencing}