%%%%%%%%%%%%%%%%%%%%%%%%%%%%%%%%%%%%%%%%%%%%%%%%%%%%
%%											  	 %%
%% Author : Andreas Apostolatos               	 %%
%%											  	 %%
%% e-mail : andreas.apostolatos@tum.de		   	 %%
%%											  	 %%
%% 01_introduction.tex					  	   	 %%
%%											  	 %%
%%%%%%%%%%%%%%%%%%%%%%%%%%%%%%%%%%%%%%%%%%%%%%%%%%%%

\section{INTRODUCTION}
The goal of the project is to build a framework for performing a sensitivity analysis for a 2-dimensional linear elasticity problem. Sensitivity analysis is an important tool in structural design, as it is a prerequisite for structural optimization.
\\[6pt]
Three distinct software tools were utilized for various aspects of the sensitivity analysis. First, GiD is used as a preprocessor, where the problem is defined. Geometry, boundary conditions, material properties, and the mesh are all selected in GiD. Moreover, as explained later in appendix \ref{section:appendix_GiD}, the GiD user interface was modified such that the user can easily set the desired parameters for the sensitivity analysis in the preprocessing stage. GiD then generates a \texttt{.dat} file that is used as an input in the subsequent stage.% add reference 
\\[6pt]
The second stage, processing, consists of Matlab code provided by the chair of Structural Analysis, with the sensitivity analysis package added for the purpose of this project. Matlab reads in the \texttt{.dat} file generated by GiD, parses it, and computes the results accordingly. More details on the implementation are provided in appendix \ref{section:appendix_matlab}. \\[6pt]
Finally, Paraview is used as a postprocessing tool, in order to visualize the computed results. As shown later, this program allows the user to have an intuitive and immediate visualization of the structure's behavior under the prescribed conditions.\\[6pt]
In this particular case, three different types of sensitivity analysis are computed: \textit{displacement}, \textit{strain energy} and \textit{Von Mises stress}.  Moreover, an algorithm for the \textit{optimization} of the structure is developed, as explained in more detail later.\\[6pt]
As discussed in the results section, the sensitivity analysis offers a useful technique to compute through a numerical approach the influence of the different parts of the structure on an observed function, be it global or specific to one single node. This allows to understand which part of the structure is more effective to change in order to affect the results on the objective functions as desired in the analyzed scenario. As an example, the case computed in the topic can be taken: by the computation, it's possible to understand where to modify the structure in order to optimize the displacement at a given node and the global strain energy of the structure. For further explanation, see section \ref{section:optimization}.
\\[6pt]
